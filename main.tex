% LaTeX Curriculum Vitae Template
%
% Copyright (C) 2004-2009 Jason Blevins <jrblevin@sdf.lonestar.org>
% http://jblevins.org/projects/cv-template/
%
% You may use use this document as a template to create your own CV
% and you may redistribute the source code freely. No attribution is
% required in any resulting documents. I do ask that you please leave
% this notice and the above URL in the source code if you choose to
% redistribute this file.

\documentclass[letterpaper]{article}

\usepackage{hyperref}
\usepackage{geometry}
\usepackage{amsmath,amsthm,amssymb}
\usepackage[utf8]{inputenc}
\usepackage[russian]{babel}
\usepackage{graphicx}
\usepackage{epsfig}
\usepackage{wrapfig}
\usepackage{wasysym}
\usepackage{stackrel} 
\usepackage[noend]{algorithmic}
\usepackage{moreverb}
\usepackage[usenames]{color}
\usepackage{colortbl}
\usepackage{tikz}

% Comment the following lines to use the default Computer Modern font
% instead of the Palatino font provided by the mathpazo package.
% Remove the 'osf' bit if you don't like the old style figures.
\usepackage[T1]{fontenc}
%\usepackage[sc,osf]{mathpazo}

% Set your name here
\def\name{Евгений Лупашин}
%\def\name{Evgeniy Lupashin}

% Replace this with a link to your CV if you like, or set it empty
% (as in \def\footerlink{}) to remove the link in the footer:
\def\footerlink{}

\let\oldhref\href
\renewcommand{\href}[2]{\oldhref{#1}{\bfseries#2}}


% The following metadata will show up in the PDF properties
\hypersetup{
  colorlinks = true,
  urlcolor = black,
  pdfauthor = {\name},
  pdfkeywords = {economics, statistics, mathematics},
  pdftitle = {\name: Curriculum Vitae},
  pdfsubject = {Curriculum Vitae},
  pdfpagemode = UseNone
}

\geometry{
  body={6.5in, 10.5in},
  left=1.0in,
  right=1.1in,
  top=0.75in
}

% Customize page headers
\pagestyle{myheadings}
\markright{\name}
\thispagestyle{empty}

% Custom section fonts
\usepackage{xcolor}
\usepackage{xcolor-patch}
\usepackage{sectsty}

\definecolor{myblue}{rgb}{0.231372549, 0.45882353, 0.705882353}
\sectionfont{\rmfamily\mdseries\Large}
\subsectionfont{\rmfamily\mdseries\itshape\large}
%\chapterfont{\color{myblue}}
\sectionfont{\color{myblue}\large\itshape\underline}

% Other possible font commands include:
% \ttfamily for teletype,
% \sffamily for sans serif,
% \bfseries for bold,
% \scshape for small caps,
% \normalsize, \large, \Large, \LARGE sizes.

% Don't indent paragraphs.
\setlength\parindent{0em}

% Make lists without bullets
%\renewenvironment{itemize}{
%  \begin{list}{}{
%    \setlength{\leftmargin}{1.5em}
%  }
%}{
%  \end{list}
%}
\renewcommand\labelitemi{\raisebox{\mylen}{\tiny$\bullet$}}

\begin{document}

% Place name at left
{\huge \textbf{\name}}

% Alternatively, print name centered and bold:
%\centerline{\huge \bf \name}

\vspace{0.1in}

\begin{minipage}{0.45\linewidth}
  \href{https://www.hse.ru/}{НИУ ВШЭ} \\
  Прикладная математика и информатика \\
  Кочновский пр. 3 \\
  Москва 125319
\end{minipage}
\begin{minipage}{0.45\linewidth}
  \begin{tabular}{ll}
    Телефон: & +7 (916) 846-4329 \\
    Email: & \href{mailto:lupasin1999@gmail.com}{\tt lupasin1999@gmail.com} \\
  \end{tabular}
\end{minipage}


\section*{Образование}
\begin{itemize}
	\item НИУ ВШЭ, Факультет компьютерных наук, группа с углублённым изучением математики и программирования, 2017-2021
	\item Зимняя Компьютерная Школа, 2017
	\item Летняя Компьютерная Школа, В' и В+, 2015-2016
\end{itemize}


\section*{Опыт}
\begin{itemize}
	\item \href{https://city4people.ru/post/kalkulyator-elektrobusov.html}{Проект} для <<Городских проектов Каца и Варламова>>
	\item Участие в хакатоне UrbanHack с проектом по анализу расходов на различные виды общественного транспорта
	\item Летняя практика в НИУ ВШЭ по C++ и Qt на тему сбора данных и анализа профессиональных качеств наборщиков текста
	\item Ссылка на GitHub: \href{https://github.com/PipeKnight}{Тык}
\end{itemize}


\section*{Навыки}
\subsection*{Уверенное владение:}
\begin{itemize}
	\item C++, Python 3
\end{itemize}
\subsection*{Базовое владение:}
\begin{itemize}
	\item Qt, Java, Assembler
\end{itemize}

\section*{Прочее}
\subsection*{Успехи в олимпиадах:}
\begin{itemize}
	\item Победитель <<Высшей пробы>> по информатике в 2016 и 2017 годах
	\item Призёр регионального этапа ВОШ по информатике в 2015-2017 годах
	\item Призёр заключительного этапа ИОИП по информатике в 2017 году
\end{itemize}
\subsection*{Пройденные курсы:}
\begin{itemize}
	\item Основы и методология программирования
	\item Алгоритмы и структуры данных
	\item Математический анализ
	\item Дискретная математика
	\item Линайная алгебра и геометрия
	\item Алгебра
	\item Теория вероятности
\end{itemize}


\end{document}